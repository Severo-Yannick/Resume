\documentclass{article}

\usepackage{titling}
\usepackage{geometry}
\usepackage{hyperref}
\usepackage{multicol}
\usepackage{enumitem}
\usepackage{tabularx}
\usepackage{cellspace}
\usepackage{booktabs}

% En tete de ton CV, le poste que tu recherches
\title{Développeur Frontend}
% Ton nom et ton prenom + ton age (optionel)
\author{Yannick SEVERO}
\date{}

\geometry{
    top=1.7cm,
    bottom=1.7cm,
    left=2cm,
    right=2cm
}
\pagestyle{empty} % Retire la pagination
\setlist[itemize]{itemsep=0pt} % Espace entre les items dans une liste
\setlength{\droptitle}{-5em} % Ajuster la valeur pour augmenter / diminuer le spacing dans le document

\begin{document}
\maketitle
\thispagestyle{empty} % Retirer le numéro de la premiere page
\vspace{-4em}

% Dans cette section, numéro de telephone, email, liens vers linkedin, mon blog et ma ville
\begin{center}
+33 7 00 00 00 00 | \href{mailto:tonmail@gmail.com}{\underline{tonmail@gmail.com}} | \href{https://linkedin.com/in/yannick-severo}{\underline{linkedin.com/in/yannick-severo}} | \href{https://www.faceaucode.com/}{\underline{blog: faceaucode}}
\end{center}
\begin{center}
\textbf{Avignon, France}
\end{center}

%%%%%%%%%%%%%%%%%%%%%%%%%%%%%%%%%%%%%%%%%%%%%%
%%%%%%%%%%%%%%%%%%%%%%%%%%%%%%%%%%%%%%%%%%%%%%
%%%%%%%%%%%%%%%%%%%%%%%%%%%%%%%%%%%%%%%%%%%%%%

% Phrase d'intro qui donne mon expertise
\begin{center}
\textbf{Développeur React \& Typescript. Une appétence pour le clean code, les tests et la clean architecture. Un grand intérêt pour l'UX et l'accessibilité. Expérience en environnements multisectoriels : Logistique / Hôtelier / BTP}
\end{center}

%%%%%%%%%%%%%%%%%%%%%%%%%%%%%%%%%%%%%%%%%%%%%%
%%%%%%%%%%%%%%%%%%%%%%%%%%%%%%%%%%%%%%%%%%%%%%
%%%%%%%%%%%%%%%%%%%%%%%%%%%%%%%%%%%%%%%%%%%%%%
% Tables des compétences
\section*{Compétences techniques}

\begin{tabularx}{\textwidth}{@{}lX@{}}

  \textbf{Programmation} & Accessibilité, Tests unitaires, Clean code, Refactoring, Code review, Documentation technique \\
  \addlinespace[5pt] % Add margin below this row

  \textbf{Langages} & Typescript \& Javascript \\
  \addlinespace[5pt] % Add margin below this row

  \textbf{Développement Web} & React, Modern state management solutions (Redux, Recoil), Design systems (Styled components, MUI) \\
  \addlinespace[5pt] % Add margin below this row

  \textbf{Développement Mobile} & React, Ionic, Cordova, Modern state management solutions (Mobx) \\
  \addlinespace[5pt] % Add margin below this row

  \textbf{Outils} & Build tools (Webpack, Vite, Xcode), Testing tools (Jest, React Testing Library, Postman), Linter tools (prettier, ESLint), Traduction (i18next, mustache), Maquettes (Figma, Zeplin), Versioning (Git, GitHub, GitLab) \\
  \addlinespace[5pt] % Add margin below this row

  \textbf{Gestion de projet} & Méthodes Agile, Jira, Confluence \\
  \addlinespace[5pt] % Add margin below this row
  
\end{tabularx}

%%%%%%%%%%%%%%%%%%%%%%%%%%%%%%%%%%%%%%%%%%%%%%
%%%%%%%%%%%%%%%%%%%%%%%%%%%%%%%%%%%%%%%%%%%%%%
%%%%%%%%%%%%%%%%%%%%%%%%%%%%%%%%%%%%%%%%%%%%%%
\vspace{4ex}
\hrulefill
\section*{Experiences}

%%%%%%%%%%%%%%%%%%%%%%%%%%%%%%%%%%%%%%%
%% 
%% EXPERIENCE 1
%%
\paragraph{Développeur frontend chez CMA-CGM}\hspace*{\fill}Avril 2023 - Aujourd'hui

\noindent
Application mobile E-commerce Multi-plateforme – iOS \& Android \\
100K+ Téléchargement - Noté 4,5/5 sur Google Play et 4,6/5 sur l'App store

\raggedright  
\begin{itemize}
  \item{Développement et refonte de l’application mobile}
  \item{Développement de nouvelles fonctionnalités et modules}
  \item{Résolution de Bugs}
  \item{Amélioration de l’interface utilisateur}
  \item{Implémentation de tests unitaires}
  \item{Cérémonies agile}
\end{itemize}
\noindent
Environnement technique: Ionic, Reactjs, MobX, Jest, TypeScript, Node.js, NPM
%% 
%% STOP EXPERIENCE 1
%%
%%%%%%%%%%%%%%%%%%%%%%%%%%%%%%%%%%%%%%%
%% 
%% EXPERIENCE 2
%%
\paragraph{Développeur React/NodeJS chez My Groom Service}\hspace*{\fill}Octobre 2021 - Février 2023

\noindent
Plateformes SAAS – Création de CMS interne - Secteur hôtelier \\
CMS permettant de créer et maintenir plusieurs centaines de sites de réservation hôtelière

\raggedright
\begin{itemize}
  \item{Développement d’outils hôteliers}
  \item{Développement de nouvelles fonctionnalités principales}
  \item{Bug fix, Amélioration des outils}
  \item{Architecture 3 tiers(SPA, API REST et DB)}
  \item{Interventions sur différents projets}
  \item{Communication entre les différentes API’s à travers la méthode SSO}
  \item{Tests API avec les librairies Jest et Supertest}
\end{itemize}
\noindent
Environnement technique: Reactjs, Redux, JavaScript, Bootstrap, NodeJS, Express, GitLab
%% 
%% STOP EXPERIENCE 2
%%
%%%%%%%%%%%%%%%%%%%%%%%%%%%%%%%%%%%%%%%
%% 
%% EXPERIENCE 3
%%
\paragraph{Développeur Front-end en stage}\hspace*{\fill}Octobre 2021 - Février 2023

\noindent
B-TIP | Développement d’une application web - Secteur alimentaire

\raggedright  
\begin{itemize}
  \item{Fonctionnalités permettant aux clients n’ayant pas l’application
mobile de passer leurs commandes depuis un navigateur web}
  \item{Développement de fonctionnalités principales}
  \item{Analyse des differents besoins utilisateur}
\end{itemize}
\noindent
Environnement technique: JavaScript, ReactJS, Bootstrap, NodeJS, Bitbucket
%% 
%% STOP EXPERIENCE 3
%%
%%%%%%%%%%%%%%%%%%%%%%%%%%%%%%%%%%%%%%%
%% 
%%%%%%%%%%%%%%%%%%%%%%%%%%%%%%%%%%%%%%%
%% 
%% EXPERIENCE 4
%%
\paragraph{Web master – Intégrateur web}\hspace*{\fill}Avril 2019 - Septembre 2021

\noindent
Lck-Lecrazykids | Développement d’une application web e-commerce - Secteur vêtements pour bébés

\raggedright  
\begin{itemize}
  \item{Conception et developpement du site e-commerce avec le CMS Wordpress}
  \item{Intégration de plus de 1500 articles}
  \item{Mise en place d’un paiement en ligne avec stripe}
  \item{Maintenance (sécurité, sauvegardes, mise à jour de plugins)}
  \item{Mise en production}
  \item{Résolution de bugs}
\end{itemize}
\noindent
Environnement technique: Wordpress, O2switch, Stripe, GIMP, HTML, CSS
%% 
%% STOP EXPERIENCE 4
%%
%%%%%%%%%%%%%%%%%%%%%%%%%%%%%%%%%%%%%%%
%% 

%%%%%%%%%%%%%%%%%%%%%%%%%%%%%%%%%%%%%%%%%%%%%%
%%%%%%%%%%%%%%%%%%%%%%%%%%%%%%%%%%%%%%%%%%%%%%
%%%%%%%%%%%%%%%%%%%%%%%%%%%%%%%%%%%%%%%%%%%%%%
\vspace{2ex}
\hrulefill
\section*{Formations}
\paragraph{Bachelor Concepteur Développeur d’Application BAC+3 école O'clock}\hspace*{\fill}Octobre 2021 - Avril 2023

\vspace{\baselineskip}
\textbf{Cours:} Développer la partie front-end et back-end d’une interface utilisateur web.
Concevoir une base de données.
Concevoir une application.
Construire une application organisée en couches.
Développer une application mobile.
Préparer et exécuter les plans de tests d’une application

\paragraph{Développeur Web Full Stack Javascript BAC+2 école O'clock}\hspace*{\fill}Février 2021 - Juillet 2023

\vspace{\baselineskip}
\textbf{Cours:} Maquetter une application.
Créer une base de données.
Développer une interface utilisateur web dynamique.
Développer la partie back-end d’une application web ou web mobile.
Réaliser une interface utilisateur web statique et adaptable.
Mettre en œuvre et élaborer des composants dans une application.

%%%%%%%%%%%%%%%%%%%%%%%%%%%%%%%%%%%%%%%%%%%%%%
%%%%%%%%%%%%%%%%%%%%%%%%%%%%%%%%%%%%%%%%%%%%%%
%%%%%%%%%%%%%%%%%%%%%%%%%%%%%%%%%%%%%%%%%%%%%%
\vspace{2ex}
\hrulefill
\section*{Certifications}

\paragraph{Fresque du numérique} | Sensibilisation à l’environnement\hspace*{\fill}Avril 2024
\paragraph{Méthodes agiles} | What Agile is, how it works, and the mindset necessary for working iteratively\hspace*{\fill}Avril 2024
\paragraph{Clean Code} | Common patterns, best practices and rules related to writing clean code\hspace*{\fill}Février 2024
\paragraph{Opquast} | Maîtrise de la qualité en projet Web - 810 points - Niveau avancé\hspace*{\fill}Aout 2021
\paragraph{CNIL} | MOOC L'Atelier RGPD\hspace*{\fill}Février 2021
\paragraph{freeCodeCamp} | Responsive Web Design\hspace*{\fill}Février 2021

\end{document}

